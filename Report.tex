\documentclass[12pt,a4paper,oneside]{scrartcl}
\usepackage[left=2.5cm,right=2.5cm,top=3cm,bottom=3cm]{geometry}

\usepackage[utf8x]{inputenc}
\usepackage[T1]{fontenc}
\usepackage[english]{babel}

\usepackage[fleqn]{mathtools} %Replaces AMSMATH

\usepackage{fancyhdr}

%Nice Tables
\usepackage{tabularx}
\usepackage{booktabs}



%Section Heading font
\setkomafont{section}{\normalfont\large\rmfamily\bfseries}
\setkomafont{subsection}{\normalfont\normalsize\rmfamily\bfseries}
\setkomafont{sectioning}{\normalfont\large\rmfamily\bfseries}

\title{Title}
\author{Name}
\date{Date}


\pagestyle{fancy}
\lhead{\footnotesize Astrophysics SH2402 Seminar - Microlensing \\ date}
\rhead{\footnotesize Linda Eliasson, Alejandro Ruiz \\ Albyn Lowe, Michael Bühlmann}


\begin{document}

\section{Introduction to microlensing}
\subsection{History}
\subsection{Theory}

\section{Applications today}
\subsection{Exoplanets}
\subsection{Dark Matter}


\section{Teaching plans}


%Need
%General intro of gravitational lensing, specially micro, specially exoplanet finding
%Most interesting aspects: What we detect (difference btwn star spectra and star with planet spectra), what has been detected
%How do we plan to teach the essence (Ehmm... slides&talk... Create curiosity, wanting to know more?)
%(1-2 pages)

\end{document}
